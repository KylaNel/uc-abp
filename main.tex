\documentclass[sigconf,nonacm,screen]{acmart}
\settopmatter{printfolios=true}
\geometry{a4paper}

%% PACKAGES %%
\usepackage[utf8]{inputenc}
\usepackage{microtype}
\usepackage[capitalise,nameinlink,noabbrev]{cleveref}
\usepackage{tikz}
\usepackage{tcolorbox}
\usepackage{amssymb}

\newcommand\extitem{\begin{tiny}$\blacksquare$ \end{tiny}}
\newcommand{\pluseq}{\mathrel{+}=}

\usepackage{lipsum} 

% % % % % % % % % % % % % % % % % % % % %
%          STUDENT INFORMATION 
% % % % % % % % % % % % % % % % % % % % %
\title{Universally Composable Anonymous Broadcast Protocols}

\author{Kyla Nel} %% ENTER YOUR NAME HERE!!!
\affiliation{
  \country{2576953n} %% ENTER YOUR MATRICULATION NUMBER HERE!!!
}


%% DOCUMENT %%

\begin{document}
% % % % % % % % % % % % % % % % % % % % %
% 			ABSTRACT
% % % % % % % % % % % % % % % % % % % % %

\begin{abstract}

\begin{tcolorbox}[enhanced, colback=white, arc=5pt, breakable]
\noindent\emph{\underline{Anonymous broadcast functionality $\mathcal{F}_R^K$}}\\[5pt]
 Initialise:
 \begin{enumerate}
     \item a list of pending messages $L_{pend} \leftarrow []$
     \item $status_P\in\{0,1\}\leftarrow 0$ for party $P$ indicating whether $P$ has sent a message in the current round
 \end{enumerate}
 
 
\extitem Upon receiving (\textbf{sid}, \textbf{WRITE}, $M$) from honest party $P$ or (\textbf{sid}, \textbf{WRITE}, $M$, $P$) from $S$ on behalf of corrupted party $P$:\\
If $status_P=0$, then
\begin{enumerate}
    \item set $status_P\leftarrow 1$
    \item append $M$ to $L_{pend}$
    \item if $|L_{pend}|=K$, then
    \begin{enumerate}
        \item order the messages lexicographically as $<M_1,...,M_K>$
        \item set $L{pend}\leftarrow []$
        \item set $status_P\leftarrow 0$ for every $P$
        \item send (\textbf{sid}, \textbf{BROADCAST}, $<M_1,...M_K>$ to all parties and (\textbf{sid}, \textbf{BROADCAST}, $<M_1,...M_K>, P)$ to $S$
    \end{enumerate}
    \item else, send (\textbf{sid}, \textbf{WRITE}, $|M|$, $P$) to $S$
\end{enumerate}





\end{tcolorbox}
\captionof{figure}{Anonymous broadcast ideal functionality.}
\label{fig:riposte_functionality}
\begin{table*}
% \begin{tcolorbox}[enhanced, colback=white, arc=5pt, breakable]




\begin{tcolorbox}[colback=white, arc=5pt]
\noindent\emph{\underline{Riposte UC Protocol}}\\[5pt]
Variables:
\begin{itemize}
    \item $R$ - number of rows in each database table
    \item $C$ - length of messages
    \item $e_{\ell,M}$ - $R$ x $C$ x 2 bitstring containing 0 everywhere except in row $l$ which contains $(M, M^2)\in \mathbb{F}^k$, where $M$ is the message to be sent
    \item $K$ - message limit in a round
\end{itemize}
Initialise:
\begin{enumerate}
    \item $status_P\in\{0,1\}\leftarrow 0$ for party $P$ indicating whether $P$ has sent a message in the current round
    \item $count\in\mathbb{N}\leftarrow 0$ indicating the number of valid write requests received this round
\end{enumerate}
\extitem Upon receiving (\textbf{sid}, \textbf{WRITE}, $M$) from $P$\\
If $status_P=0$, then
\begin{enumerate}
    \item set $status_P\leftarrow 1$
    \item $P$ chooses index $l \overset{{\scriptscriptstyle\$}}{\leftarrow} [0,R)$ and generates bitstring $e_l$
    \item $P$ generates random $R$ x $C$ x 2 bitstring $r$
    \item $P$ sends (\textbf{prove}, $P$, $e_{\ell,M}$) to $\mathcal{F}_{ZK}^{R,R'}$
    \item $P$ sends $r\oplus e_{\ell,M}$ to Server B using $\mathcal{F}_{\mathcal{AEC}}(\{A,B\})$
    \item $P$ sends $r$ to Server A using $\mathcal{F}_{\mathcal{AEC}}(\{A,B\})$
    
    \item $count \pluseq 1$
    \item if $count=K$, then
    \begin{enumerate}
        \item set $status_p\leftarrow0$
        \item set $count \leftarrow 0$
    \end{enumerate}
\end{enumerate}

% \extitem Upon receiving (\textbf{sid}, \textbf{BROADCAST}, $M_A$) from Server A and (\textbf{sid}, \textbf{BROADCAST}, $M_B$) from Server B
% \begin{enumerate}
%     \item Verify that $M_A = M_B$
%     \item If $M_A = M_B$, forward to $\mathcal{Z}$
% \end{enumerate}

\extitem Upon receiving (\textbf{sid}, \textbf{SEND}, $r\oplus e_l$) from $P$, if $P$ has not executed a write request in this phase, then
    Server B executes the following:
    
    
    Upon receiving (\textbf{proof}, l(y)) from $\mathcal{F}_{ZK}^{R,R'}$, if received (\textbf{sid}, WRITE, M) from $P$:
    \begin{enumerate}
        \item add $r\oplus e_{\ell,M}$ into its database  
        \item if $count=K$, then
        \item 
        \begin{enumerate}
            \item add $r\oplus e_l$ into its database  
            \item if $count=K$, then
            \begin{enumerate}
                \item combine database with Server A's database
                \item resolve collisions
                \item order messages lexicographically as $M_B=<M_1,...,M_K>$
                \item broadcast messages to all parties
            \end{enumerate}
        \end{enumerate}
    \end{enumerate}
    Upon receiving (\textbf{sid}, WRITE, M) from $P$, if received (\textbf{proof}, l(y)) from $\mathcal{F}_{ZK}^{R,R'}$:
        \begin{enumerate}
            \item add $r\oplus e_l$ into its database  
            \item if $count=K$, then
            \begin{enumerate}
                \item combine database with Server A's database
                \item check for collisions
                \item resolve collisions
                \item order messages lexicographically as $M_B=<M_1,...,M_K>$
                \item broadcast messages to all parties
            \end{enumerate}
        \end{enumerate}


\extitem Upon receiving (\textbf{sid}, \textbf{SEND}, $r$) from $P$, if $P$ has not executed a write request in this phase, then
    Server A executes the following:
    Upon receiving (\textbf{proof}, l(y)) from $\mathcal{F}_{ZK}^{R,R'}$, if received (\textbf{sid}, WRITE, M) from $P$:
        \begin{enumerate}
            \item add $r$ into its database
            \item if $count=K$, then
            \begin{enumerate}
                \item combine database with Server B's database
                \item resolve collisions
                \item order messages lexicographically as $M_A=<M_1,...,M_K>$
                \item broadcast messages to all parties
            \end{enumerate}
        \end{enumerate}
    Upon receiving (\textbf{sid}, WRITE, M) from $P$, if received (\textbf{proof}, l(y)) from $\mathcal{F}_{ZK}^{R,R'}$:
        \begin{enumerate}
            \item XOR $r$ into its database
            \item if $count=K$, then
            \begin{enumerate}
                \item combine database with Server B's database
                \item resolve collisions
                \item order messages lexicographically as $M_A=<M_1,...,M_K>$
                \item broadcast messages to all parties
            \end{enumerate}
        \end{enumerate}
\end{tcolorbox}

\captionof{figure}{Anonymous broadcast protocol.}
\label{fig:riposte_protocol}
\end{table*}
% \begin{tcolorbox}[enhanced, colback=white, arc=5pt, breakable]
\begin{tcolorbox}[colback=white, arc=5pt]
\noindent\emph{\underline{AE channel functionality $\mathcal{F}_{\mathcal{AEC}}(\{A,B\})$}}\\[5pt]
 Initialise a list $PendingMsg\leftarrow\emptyset$.
 
\extitem Upon receiving (\textbf{sid}, \textbf{SEND}, $M$) from P, if P is honest, then:\\
\begin{enumerate}
    \item If $\{A,B\} \setminus \{P\}$ is corrupted, then send (\textbf{sid}, \textbf{SEND}, $M$, P) to $\mathcal{S}$.
    \item  If $\{A,B\} \setminus \{P\}$ is honest, then
    \begin{itemize}
        \item Choose a random tag $\overset{\$}{\leftarrow}\{0,1\}^\lambda $.
        \item Add $(\textbf{tag}, M, P)$ to $PendingMsg$
        \item Send (\textbf{sid}, \textbf{SEND}, \textbf{tag}, $|M|$, P, \{A,B\} $\setminus$ \{P\}) to $\mathcal{S}$.
    \end{itemize}
    \item Upon receiving (\textbf{sid}, \textbf{ALLOW}, \textbf{tag}) from $\mathcal{S}$, if there is a (\textbf{tag}, $M$, P) in $PendingMsg$, then remove (\textbf{tag}, $M$, P) from $PendingMsg$ and send (\textbf{sid},\textbf{SEND},$M$) to \{A,B\}$\setminus$\{P\}
\end{enumerate}

\end{tcolorbox}
\captionof{figure}{Anonymous broadcast ideal functionality.}
\label{fig:ae_functionality}
% \begin{tcolorbox}[enhanced, colback=white, arc=5pt, breakable]
\begin{tcolorbox}[colback=white, arc=5pt]
    \noindent\emph{\underline{Non-interactive zero knowledge functionality $\mathcal{F}_{NIZK}^{R}$}}\\[5pt]
    \begin{enumerate}
        \item \textbf{Proof:} On input (\text{prove}, \textbf{sid}, $x$, $w$) from party $P$: if $\mathcal{R}(x,w)=1$ then send (\text{prove}, $P$, \textbf{sid}, $x$) to $\mathcal{S}$. Upon receiving (\text{proof}, \textbf{sid},$\pi$) from $\mathcal{S}$, store (\textbf{sid}, $x$, $w$, $\pi$) and send (\text{proof}, \textbf{sid},$\pi$) to $P$.
        \item \textbf{Verification:} On input (\text{verify}, \textbf{sid},$x$,$\pi$) from a party $V$: If (\textbf{sid},$x$,$w$,$\pi$) is stored, then return (\text{verification}, \textbf{sid}, $x$, $\pi$, $\mathcal{R}(x,w)$) to $V$. Else, send (\text{verify}, $V$, \textbf{sid}, $x$, $\pi$) to $\mathcal{S}$. Upon receiving, (\text{witness}, $\textbf{sid}$, $w$) from $\mathcal{S}$, store (\textbf{sid}, $x$, $w$, $\pi$) and return (\text{verification}, \textbf{sid}, $x$, $\pi$, $\mathcal{R}(x,w)$) to $V$.
        \item \textbf{Corruption:} When receiving (\text{corrupt}, \textbf{sid}) from $\mathcal{S}$, mark \textbf{sid} as corrupted. If there is a stored tuple (\textbf{sid}, $x$, $w$, $\pi$), then send it to $\mathcal{S}$.
    \end{enumerate}
\end{tcolorbox}
\captionof{figure}{Non-interactive zero knowledge functionality.}
\label{fig:zk_functionality}
% \begin{tcolorbox}[enhanced, colback=white, arc=5pt, breakable]
\begin{tcolorbox}[colback=white, arc=5pt]
    \noindent\emph{\underline{Broadcast functionality $\mathcal{F}_{BC}$}}\\[5pt]
    The functionality interacts with an adversary $\mathcal{S}$ and a set $\mathcal{P}=\{P_1,...,P_n\}$ of parties.
    \begin{itemize}
        \item Upon receiving (\textbf{sid}, \textbf{BROADCAST}, $M$) from $P_i$, send (\textbf{sid}, \textbf{BROADCAST}, $M$) to all parties in $\mathcal{P}$ and $\mathcal{S}$.
    \end{itemize}
    
    \end{tcolorbox}
    \captionof{figure}{Broadcast functionality $\mathcal{F}_{BC}$}
    \label{fig:bc_functionality}
\end{abstract}

%%%%%%%%%%%% DO NOT EDIT THIS PART!!! %%%%%%%%%%%%
%%%%%%%%%%%%%%%%%%%%%%%%%%%%%%%%%%%%%%%%%%%%%%%%%%
\maketitle
\tikz [remember picture, overlay] %
\node [shift={(0.5cm,-0.5cm)}] at (current page.north west) %
[anchor=north west,scale=0.7] %
{\includegraphics{CompSci_logo.pdf}};
%%%%%%%%%%%%%%%%%%%%%%%%%%%%%%%%%%%%%%%%%%%%%%%%%%
%%%%%%%%%%%%%%%%%%%%%%%%%%%%%%%%%%%%%%%%%%%%%%%%%%

% % % % % % % % % % % % % % % % % % % % %
% 			INTRODUCTION
% % % % % % % % % % % % % % % % % % % % %
\section{Introduction}
\label{sec:intro}

\section{Proof}

% Cases:
% \begin{enumerate}
%   \item U.r. (\textbf{sid}, WRITE, $|M|$, $P$) from functionality:
%   \begin{itemize}
%     \item Simulate a WRITE request on behalf of $P$ where $M_{Dummy}$ is all-zeroes
%     \item Generate $e_{\ell, M}$
%     \item $\mathcal{F}_{ZK}$ leaks nothing. $\mathcal{F}_{AEC}$ leaks the length of the message $|M|$, so the simulator sends $|M|$ to the adversary
%   \end{itemize}
%   \item U.r. $<M_1,...,M_k>$ from the functionality:
%   If Server A is corrupted, then
%   \begin{itemize}
%     \item Simulator sends a dummy message containing all zeroes over $\mathcal{F}_{AEC}$
%     \item Extract the subset of honest messages from the set of all messages.
%     \item Randomly assign honest messages to honest parties
%     \item Generate $e_{\ell,M}$ of the corresponding party and send $r$ to Server A, then $e_{\ell,M}\oplus r$ is the share of party $P$ for Server B
%   \end{itemize}
%   If Server B is corrupted, then
%   \begin{itemize}
%     \item Simulator equivocates by sending any $r$ to Server A
%     \item Extract the subset of honest messages from the set of all messages.
%     \item Randomly assign honest messages to honest parties
%     \item Construct a consistent $e_{\ell,M}$
%     \item Send $e_{\ell,M}\oplus r$ of the corresponding party to Server B
%   \end{itemize}
%   \item If an adversarial message is received, then 
%   \begin{itemize}
%     \item Simulator reconstructs M since it controls $\mathcal{F}_{AEC}$ and every $P$ is forced to send messages across the authenticated encrypted channel because all adversaries are passive
%     \item Tell the functionality to add M to the list of pending messages
%   \end{itemize}
% \end{enumerate}

\textbf{Case 1:} We first examine the case where Server A is corrupted and $\mathcal{S}$ controls the honest Server B. Here $\mathcal{A}$ requests to corrupt Server A in the hybrid world and so S corrupts Server A in the ideal world. Upon receiving (\textbf{sid}, \textbf{WRITE}, |$M$|, $P$) from $\mathcal{F}_R^K$ where $P$ is an honest party, $\mathcal{S}$ simulates a \textbf{WRITE} request with a dummy all-zero message $M_0=0^{|M|}$ on behalf of $P$. In this case, $\mathcal{S}$ simulates the protocol as written, using the dummy message $M_0$ to generate $e_{\ell, M_0}$. LEAK DATA. Upon receiving (\textbf{sid}, \textbf{WRITE}, |$M^*$|, $P^*$) from $\mathcal{A}$ where $P^*$ is a corrupted party, $\mathcal{S}$ reconstructs $M^*$ from the two shares sent to Server A and Server B respectively across $\mathcal{F}_{AEC}(\{A,B\})$. This is made possible by the fact that parties are passively corrupted and so are forced to communicate using $\mathcal{F}_{AEC}$ exclusively, which $\mathcal{S}$ controls. Having reconstructed $M^*$, $\mathcal{S}$ then sends a WRITE request (\textbf{SID}, \textbf{WRITE}, $M^*$, $P^*$) to the functionality. Upon receiving (\textbf{sid}, \textbf{BROADCAST}, $<M_1,...,M_K>$) from $\mathcal{F}_R^K$, $\mathcal{S}$ extracts the subset of honest messages from $<M_1,...,M_K>$ and randomly assigns the honest messages to honest parties. For each honest party $P_i$ with associated message $M_i$, $\mathcal{S}$ constructs $e_{\ell,M_i}$ and creates the share $e_{\ell,M_i} \oplus r$. $\mathcal{S}$ sends $r$ to Server A and $e_{\ell, M_i}$ to Server B. At this stage, $\mathcal{S}$ must also simulate the WRITE request sent by the party $P_K$ which is associated with $M_K$, the $K$-th message added to $L_{pend}$. This follows the steps outlined above, depending on whether $P_K$ is honest or corrupted.

\textbf{Case 2:} Next, we examine the case where Server B is corrupted and $\mathcal{S}$ controls the honest Server A. In this case, $\mathcal{S}$ follows the same steps as in Case 1, however, $\mathcal{S}$ equivocates by sending any random bitsring $r$ to Server B. After the assignment of honest messages to honest partiews, $\mathcal{S}$ constructs a consistent $e_{\ell, M}$ for every honest message $M$ and sneds $e_{\ell, M}\oplus r$ to Server B.

In both cases, whatever is sent by the corrupted server when the servers combine their states, the results will be statistically indistuinguishable from the result if both servers were honest.

\section{Collision Handling}
We assume the messages being written to the database are elements in some field $\mathbb{F}$ of prime charactersitic $p$. When a message is written to the database, it is added to the current database state by addition in $\mathbb{F}$. When a party $P_A$ wants to write the message $M_A\in \mathbb{F}$ to row $\ell$ in the database, the party actually writes the pair $(M_A,M_A^2)\in \mathbb{F}^2$ into row $\ell$. If no collision occurs, i.e. only one client writes their message to row $\ell$ in the database, recovering the original message is trivial. It is easy to check when this is the case, since the second coordinate will be the square of the first. In the case of a two-way collision, say party $P_B$ also writes their message pair $(M_B,M_B^2)\in \mathbb{F}^2$ to the same row $\ell$ in the database. Now the message pair contained in row $\ell$ of the database is $$S_1=M_A+M_B (\text{mod} p)$$ and $$S_2=M_A^2+M_B^2 (\text{mod}p)$$. Observing that $$2S_2-S_1^2=(M_A-M_B)^2(\text{mod}p)$$ we can obtain $M_A-M_B$ by taking a square root modulo $p$. Using $S_1=M_A+M_B$we can easily recover $M_A$ and $M_B$.

This way of handling two-way collisions can be generalised to $k$-way collision for $k>2$. Each party $P_i$ wanting to write their message $M_i$ into the database instead writes $(M_i,M_i^2,...,M_i^k)\in \mathbb{F}^k$ to its chosen row $\ell$. Now, if a $k$-way collision occurs, we have $k$ equations in $k$ variables which can be solved efficiently to recover all $k$ messages, as long as the characteristic $p$ of $\mathbb{F}$ is greater than $k$.

% % % % % % % % % % % % % % % % % % % % %
% 	         BACKGROUND
% % % % % % % % % % % % % % % % % % % % %
\section{Background}
\label{sec:background}


Perhaps you want to cite the seminal paper of \citet{Turing1937}, or
prior~\cite{Goedel1931} and concurrent~\cite{Church1936} work.
% % % % % % % % % % % % % % % % % % % % %
% 			YOUR SYSTEM
% % % % % % % % % % % % % % % % % % % % %
\section{My Amazing System}
\label{sec:system}


% % % % % % % % % % % % % % % % % % % % %
% 			EVALUATION
% % % % % % % % % % % % % % % % % % % % %
\section{Evaluation}
\label{sec:evaluation}

\subsection{Experimental Setup}


\subsection{Experimental Analysis}


% Our results are summarized in~\cref{tab:table1}, and a visual representation of
% our analysis can be seen in~\cref{fig:alice}.

% %% Example of table
% \begin{table}
% \footnotesize
% \begin{tabular}{lll}
% \toprule
%          & machine A                   & machine B                           \\
% \midrule
% CPU      & Intel Core i7-9700 CPU      & 2x Intel Xeon E5-2630 v3            \\
% CPU Frequency& 3.00GHz                     & 2.40GHz                             \\
% RAM      & 16GB DDR4                   & 128GB                               \\
% OS       & Ubuntu 20.04 LTS            & Ubuntu 16.04 LTS                    \\
% Compiler & GCC 9.3                     & GCC 7.3                             \\
% libm     & v2.31                       & v2.23                               \\
% libomp   & v4.5                        & v4.5                                \\
% \bottomrule
% \end{tabular}
% \caption{This is the table caption.}
% \label{tab:table1}
% \end{table}

%% Example of figure
% \begin{figure}
%     \centering
%     \includegraphics[width=0.8\columnwidth]{alice.pdf}
%     \caption{This is the figure caption.}
%     \label{fig:alice}
% \end{figure}




% % % % % % % % % % % % % % % % % % % % %
% 			CONCLUSIONS
% % % % % % % % % % % % % % % % % % % % %
\section{Conclusions}
\label{sec:conclusions}
\lipsum[1]




% % % % % % % % % % % % % % % % % % % % %
% 			ACKNOWLEDGMENTS
% % % % % % % % % % % % % % % % % % % % %
\section*{Acknowledgments}
I would like to thank \ldots

% % % % % % % % % % % % % % % % % % % % %
% 			BIBLIOGRAPHY
% % % % % % % % % % % % % % % % % % % % %
\bibliographystyle{ACM-Reference-Format}
\bibliography{refs}
\end{document}
