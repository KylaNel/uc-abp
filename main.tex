\documentclass[sigconf,nonacm,screen]{acmart}
\settopmatter{printfolios=true}
\geometry{a4paper}

%% PACKAGES %%
\usepackage[utf8]{inputenc}
\usepackage{microtype}
\usepackage[capitalise,nameinlink,noabbrev]{cleveref}
\usepackage{tikz}
\usepackage{tcolorbox}
\usepackage{amssymb}
\newcommand\extitem{\begin{tiny}$\blacksquare$ \end{tiny}}

\usepackage{lipsum} 

% % % % % % % % % % % % % % % % % % % % %
%          STUDENT INFORMATION 
% % % % % % % % % % % % % % % % % % % % %
\title{The Title of Your Paper}

\author{Clever Student} %% ENTER YOUR NAME HERE!!!
\affiliation{
  \country{1234567a} %% ENTER YOUR MATRICULATION NUMBER HERE!!!
}


%% DOCUMENT %%

\begin{document}
% % % % % % % % % % % % % % % % % % % % %
% 			ABSTRACT
% % % % % % % % % % % % % % % % % % % % %

\begin{abstract}
\begin{tcolorbox}[enhanced, colback=white, arc=5pt, breakable]
\noindent\emph{\underline{Anonymous broadcast functionality $\mathcal{F}_R^K$}}\\[5pt]
 Initialise:
 \begin{enumerate}
     \item a list of pending messages $L_{pend} \leftarrow []$
     \item $status_P\in\{0,1\}\leftarrow 0$ for party $P$ indicating whether $P$ has sent a message in the current round
 \end{enumerate}
 
 
\extitem Upon receiving (\textbf{sid}, \textbf{WRITE}, $M$) from honest party $P$ or (\textbf{sid}, \textbf{WRITE}, $M$, $P$) from $S$ on behalf of corrupted party $P$:\\
If $status_P=0$, then
\begin{enumerate}
    \item set $status_P\leftarrow 1$
    \item append $M$ to $L_{pend}$
    \item if $|L_{pend}|=K$, then
    \begin{enumerate}
        \item order the messages lexicographically as $<M_1,...,M_K>$
        \item set $L{pend}\leftarrow []$
        \item set $status_P\leftarrow 0$ for every $P$
        \item send (\textbf{sid}, \textbf{BROADCAST}, $<M_1,...M_K>$ to all parties and (\textbf{sid}, \textbf{BROADCAST}, $<M_1,...M_K>, P)$ to $S$
    \end{enumerate}
    \item else, send (\textbf{sid}, \textbf{WRITE}, $|M|$, $P$) to $S$
\end{enumerate}





\end{tcolorbox}
\captionof{figure}{Anonymous broadcast ideal functionality.}
\label{fig:riposte_functionality}
% \begin{tcolorbox}[enhanced, colback=white, arc=5pt, breakable]




\begin{tcolorbox}[colback=white, arc=5pt]
\noindent\emph{\underline{Riposte UC Protocol}}\\[5pt]
Variables:
\begin{itemize}
    \item $R$ - number of rows in each database table
    \item $C$ - length of messages
    \item $e_{\ell,M}$ - $R$ x $C$ x 2 bitstring containing 0 everywhere except in row $l$ which contains $(M, M^2)\in \mathbb{F}^k$, where $M$ is the message to be sent
    \item $K$ - message limit in a round
\end{itemize}
Initialise:
\begin{enumerate}
    \item $status_P\in\{0,1\}\leftarrow 0$ for party $P$ indicating whether $P$ has sent a message in the current round
    \item $count\in\mathbb{N}\leftarrow 0$ indicating the number of valid write requests received this round
\end{enumerate}
\extitem Upon receiving (\textbf{sid}, \textbf{WRITE}, $M$) from $P$\\
If $status_P=0$, then
\begin{enumerate}
    \item set $status_P\leftarrow 1$
    \item $P$ chooses index $l \overset{{\scriptscriptstyle\$}}{\leftarrow} [0,R)$ and generates bitstring $e_l$
    \item $P$ generates random $R$ x $C$ x 2 bitstring $r$
    \item $P$ sends (\textbf{prove}, $P$, $e_{\ell,M}$) to $\mathcal{F}_{ZK}^{R,R'}$
    \item $P$ sends $r\oplus e_{\ell,M}$ to Server B using $\mathcal{F}_{\mathcal{AEC}}(\{A,B\})$
    \item $P$ sends $r$ to Server A using $\mathcal{F}_{\mathcal{AEC}}(\{A,B\})$
    
    \item $count \pluseq 1$
    \item if $count=K$, then
    \begin{enumerate}
        \item set $status_p\leftarrow0$
        \item set $count \leftarrow 0$
    \end{enumerate}
\end{enumerate}

% \extitem Upon receiving (\textbf{sid}, \textbf{BROADCAST}, $M_A$) from Server A and (\textbf{sid}, \textbf{BROADCAST}, $M_B$) from Server B
% \begin{enumerate}
%     \item Verify that $M_A = M_B$
%     \item If $M_A = M_B$, forward to $\mathcal{Z}$
% \end{enumerate}

\extitem Upon receiving (\textbf{sid}, \textbf{SEND}, $r\oplus e_l$) from $P$, if $P$ has not executed a write request in this phase, then
    Server B executes the following:
    
    
    Upon receiving (\textbf{proof}, l(y)) from $\mathcal{F}_{ZK}^{R,R'}$, if received (\textbf{sid}, WRITE, M) from $P$:
    \begin{enumerate}
        \item add $r\oplus e_{\ell,M}$ into its database  
        \item if $count=K$, then
        \item 
        \begin{enumerate}
            \item add $r\oplus e_l$ into its database  
            \item if $count=K$, then
            \begin{enumerate}
                \item combine database with Server A's database
                \item resolve collisions
                \item order messages lexicographically as $M_B=<M_1,...,M_K>$
                \item broadcast messages to all parties
            \end{enumerate}
        \end{enumerate}
    \end{enumerate}
    Upon receiving (\textbf{sid}, WRITE, M) from $P$, if received (\textbf{proof}, l(y)) from $\mathcal{F}_{ZK}^{R,R'}$:
        \begin{enumerate}
            \item add $r\oplus e_l$ into its database  
            \item if $count=K$, then
            \begin{enumerate}
                \item combine database with Server A's database
                \item check for collisions
                \item resolve collisions
                \item order messages lexicographically as $M_B=<M_1,...,M_K>$
                \item broadcast messages to all parties
            \end{enumerate}
        \end{enumerate}


\extitem Upon receiving (\textbf{sid}, \textbf{SEND}, $r$) from $P$, if $P$ has not executed a write request in this phase, then
    Server A executes the following:
    Upon receiving (\textbf{proof}, l(y)) from $\mathcal{F}_{ZK}^{R,R'}$, if received (\textbf{sid}, WRITE, M) from $P$:
        \begin{enumerate}
            \item add $r$ into its database
            \item if $count=K$, then
            \begin{enumerate}
                \item combine database with Server B's database
                \item resolve collisions
                \item order messages lexicographically as $M_A=<M_1,...,M_K>$
                \item broadcast messages to all parties
            \end{enumerate}
        \end{enumerate}
    Upon receiving (\textbf{sid}, WRITE, M) from $P$, if received (\textbf{proof}, l(y)) from $\mathcal{F}_{ZK}^{R,R'}$:
        \begin{enumerate}
            \item XOR $r$ into its database
            \item if $count=K$, then
            \begin{enumerate}
                \item combine database with Server B's database
                \item resolve collisions
                \item order messages lexicographically as $M_A=<M_1,...,M_K>$
                \item broadcast messages to all parties
            \end{enumerate}
        \end{enumerate}
\end{tcolorbox}

\captionof{figure}{Anonymous broadcast protocol.}
\label{fig:riposte_protocol}
% \begin{tcolorbox}[enhanced, colback=white, arc=5pt, breakable]
\begin{tcolorbox}[colback=white, arc=5pt]
\noindent\emph{\underline{AE channel functionality $\mathcal{F}_{\mathcal{AEC}}(\{A,B\})$}}\\[5pt]
 Initialise a list $PendingMsg\leftarrow\emptyset$.
 
\extitem Upon receiving (\textbf{sid}, \textbf{SEND}, $M$) from P, if P is honest, then:\\
\begin{enumerate}
    \item If $\{A,B\} \setminus \{P\}$ is corrupted, then send (\textbf{sid}, \textbf{SEND}, $M$, P) to $\mathcal{S}$.
    \item  If $\{A,B\} \setminus \{P\}$ is honest, then
    \begin{itemize}
        \item Choose a random tag $\overset{\$}{\leftarrow}\{0,1\}^\lambda $.
        \item Add $(\textbf{tag}, M, P)$ to $PendingMsg$
        \item Send (\textbf{sid}, \textbf{SEND}, \textbf{tag}, $|M|$, P, \{A,B\} $\setminus$ \{P\}) to $\mathcal{S}$.
    \end{itemize}
    \item Upon receiving (\textbf{sid}, \textbf{ALLOW}, \textbf{tag}) from $\mathcal{S}$, if there is a (\textbf{tag}, $M$, P) in $PendingMsg$, then remove (\textbf{tag}, $M$, P) from $PendingMsg$ and send (\textbf{sid},\textbf{SEND},$M$) to \{A,B\}$\setminus$\{P\}
\end{enumerate}

\end{tcolorbox}
\captionof{figure}{Anonymous broadcast ideal functionality.}
\label{fig:ae_functionality}
\end{abstract}

%%%%%%%%%%%% DO NOT EDIT THIS PART!!! %%%%%%%%%%%%
%%%%%%%%%%%%%%%%%%%%%%%%%%%%%%%%%%%%%%%%%%%%%%%%%%
\maketitle
\tikz [remember picture, overlay] %
\node [shift={(0.5cm,-0.5cm)}] at (current page.north west) %
[anchor=north west,scale=0.7] %
{\includegraphics{CompSci_logo.pdf}};
%%%%%%%%%%%%%%%%%%%%%%%%%%%%%%%%%%%%%%%%%%%%%%%%%%
%%%%%%%%%%%%%%%%%%%%%%%%%%%%%%%%%%%%%%%%%%%%%%%%%%

% % % % % % % % % % % % % % % % % % % % %
% 			INTRODUCTION
% % % % % % % % % % % % % % % % % % % % %
\section{Introduction}
\label{sec:intro}

This document is the \LaTeX template for submitting your final MSci project
paper, at the School of Computing Science of the University of Glasgow. This is
an updated version, starting for the academic year 2024/25. Please make sure to
update your personal details, including title, name, and matriculation number,
within the \verb|%% STUDENT INFORMATION %%| section of the attached \LaTeX
source file \verb|main.tex|.

This template is directly derived from ACM's
\href{https://ctan.org/pkg/acmart?lang=en}{\texttt{acmart}} package; make sure
to consult the corresponding
\href{http://mirrors.ctan.org/macros/latex/contrib/acmart/acmart.pdf}{documentation}
if you face any technical issues, or if you want to explore the full range of
features offered.

Unless you are already an experienced \LaTeX user, perhaps the most
straightforward way to typeset your paper is to work directly on Overleaf; this
is a cloud service (so no need for a local installation) which you can easily
sign up for using your \verb|@glasgow.ac.uk| email. Here is also a very useful
\href{https://www.overleaf.com/learn/latex/Tutorials}{\LaTeX\ tutorial}.

% % % % % % % % % % % % % % % % % % % % %
% 	         BACKGROUND
% % % % % % % % % % % % % % % % % % % % %
\section{Background}
\label{sec:background}
\lipsum[1-2]

Perhaps you want to cite the seminal paper of \citet{Turing1937}, or
prior~\cite{Goedel1931} and concurrent~\cite{Church1936} work.
% % % % % % % % % % % % % % % % % % % % %
% 			YOUR SYSTEM
% % % % % % % % % % % % % % % % % % % % %
\section{My Amazing System}
\label{sec:system}
\lipsum[1-3]

% % % % % % % % % % % % % % % % % % % % %
% 			EVALUATION
% % % % % % % % % % % % % % % % % % % % %
\section{Evaluation}
\label{sec:evaluation}
\lipsum[1]

\subsection{Experimental Setup}
\lipsum[1-2]

\subsection{Experimental Analysis}
\lipsum[1-2]

Our results are summarized in~\cref{tab:table1}, and a visual representation of
our analysis can be seen in~\cref{fig:alice}.

%% Example of table
\begin{table}
\footnotesize
\begin{tabular}{lll}
\toprule
         & machine A                   & machine B                           \\
\midrule
CPU      & Intel Core i7-9700 CPU      & 2x Intel Xeon E5-2630 v3            \\
CPU Frequency& 3.00GHz                     & 2.40GHz                             \\
RAM      & 16GB DDR4                   & 128GB                               \\
OS       & Ubuntu 20.04 LTS            & Ubuntu 16.04 LTS                    \\
Compiler & GCC 9.3                     & GCC 7.3                             \\
libm     & v2.31                       & v2.23                               \\
libomp   & v4.5                        & v4.5                                \\
\bottomrule
\end{tabular}
\caption{This is the table caption.}
\label{tab:table1}
\end{table}

%% Example of figure
\begin{figure}
    \centering
    \includegraphics[width=0.8\columnwidth]{alice.pdf}
    \caption{This is the figure caption.}
    \label{fig:alice}
\end{figure}




% % % % % % % % % % % % % % % % % % % % %
% 			CONCLUSIONS
% % % % % % % % % % % % % % % % % % % % %
\section{Conclusions}
\label{sec:conclusions}
\lipsum[1]




% % % % % % % % % % % % % % % % % % % % %
% 			ACKNOWLEDGMENTS
% % % % % % % % % % % % % % % % % % % % %
\section*{Acknowledgments}
I would like to thank \ldots

% % % % % % % % % % % % % % % % % % % % %
% 			BIBLIOGRAPHY
% % % % % % % % % % % % % % % % % % % % %
\bibliographystyle{ACM-Reference-Format}
\bibliography{refs}
\end{document}
