\documentclass[sigconf,nonacm,screen]{acmart}
\settopmatter{printfolios=true}
\geometry{a4paper}

%% PACKAGES %%
\usepackage[utf8]{inputenc}
\usepackage{microtype}
\usepackage[capitalise,nameinlink,noabbrev]{cleveref}
\usepackage{tikz}
\usepackage{tcolorbox}
\usepackage{amssymb}

\newcommand\extitem{\begin{tiny}$\blacksquare$ \end{tiny}}
\newcommand{\pluseq}{\mathrel{+}=}

\usepackage{lipsum} 

% % % % % % % % % % % % % % % % % % % % %
%          STUDENT INFORMATION 
% % % % % % % % % % % % % % % % % % % % %
\title{The Title of Your Paper}

\author{Kyla Nel} %% ENTER YOUR NAME HERE!!!
\affiliation{
  \country{2576953n} %% ENTER YOUR MATRICULATION NUMBER HERE!!!
}


%% DOCUMENT %%

\begin{document}
% % % % % % % % % % % % % % % % % % % % %
% 			ABSTRACT
% % % % % % % % % % % % % % % % % % % % %

\begin{abstract}

\begin{tcolorbox}[enhanced, colback=white, arc=5pt, breakable]
\noindent\emph{\underline{Anonymous broadcast functionality $\mathcal{F}_R^K$}}\\[5pt]
 Initialise:
 \begin{enumerate}
     \item a list of pending messages $L_{pend} \leftarrow []$
     \item $status_P\in\{0,1\}\leftarrow 0$ for party $P$ indicating whether $P$ has sent a message in the current round
 \end{enumerate}
 
 
\extitem Upon receiving (\textbf{sid}, \textbf{WRITE}, $M$) from honest party $P$ or (\textbf{sid}, \textbf{WRITE}, $M$, $P$) from $S$ on behalf of corrupted party $P$:\\
If $status_P=0$, then
\begin{enumerate}
    \item set $status_P\leftarrow 1$
    \item append $M$ to $L_{pend}$
    \item if $|L_{pend}|=K$, then
    \begin{enumerate}
        \item order the messages lexicographically as $<M_1,...,M_K>$
        \item set $L{pend}\leftarrow []$
        \item set $status_P\leftarrow 0$ for every $P$
        \item send (\textbf{sid}, \textbf{BROADCAST}, $<M_1,...M_K>$ to all parties and (\textbf{sid}, \textbf{BROADCAST}, $<M_1,...M_K>, P)$ to $S$
    \end{enumerate}
    \item else, send (\textbf{sid}, \textbf{WRITE}, $|M|$, $P$) to $S$
\end{enumerate}





\end{tcolorbox}
\captionof{figure}{Anonymous broadcast ideal functionality.}
\label{fig:riposte_functionality}
% \begin{tcolorbox}[enhanced, colback=white, arc=5pt, breakable]




\begin{tcolorbox}[colback=white, arc=5pt]
\noindent\emph{\underline{Riposte UC Protocol}}\\[5pt]
Variables:
\begin{itemize}
    \item $R$ - number of rows in each database table
    \item $C$ - length of messages
    \item $e_{\ell,M}$ - $R$ x $C$ x 2 bitstring containing 0 everywhere except in row $l$ which contains $(M, M^2)\in \mathbb{F}^k$, where $M$ is the message to be sent
    \item $K$ - message limit in a round
\end{itemize}
Initialise:
\begin{enumerate}
    \item $status_P\in\{0,1\}\leftarrow 0$ for party $P$ indicating whether $P$ has sent a message in the current round
    \item $count\in\mathbb{N}\leftarrow 0$ indicating the number of valid write requests received this round
\end{enumerate}
\extitem Upon receiving (\textbf{sid}, \textbf{WRITE}, $M$) from $P$\\
If $status_P=0$, then
\begin{enumerate}
    \item set $status_P\leftarrow 1$
    \item $P$ chooses index $l \overset{{\scriptscriptstyle\$}}{\leftarrow} [0,R)$ and generates bitstring $e_l$
    \item $P$ generates random $R$ x $C$ x 2 bitstring $r$
    \item $P$ sends (\textbf{prove}, $P$, $e_{\ell,M}$) to $\mathcal{F}_{ZK}^{R,R'}$
    \item $P$ sends $r\oplus e_{\ell,M}$ to Server B using $\mathcal{F}_{\mathcal{AEC}}(\{A,B\})$
    \item $P$ sends $r$ to Server A using $\mathcal{F}_{\mathcal{AEC}}(\{A,B\})$
    
    \item $count \pluseq 1$
    \item if $count=K$, then
    \begin{enumerate}
        \item set $status_p\leftarrow0$
        \item set $count \leftarrow 0$
    \end{enumerate}
\end{enumerate}

% \extitem Upon receiving (\textbf{sid}, \textbf{BROADCAST}, $M_A$) from Server A and (\textbf{sid}, \textbf{BROADCAST}, $M_B$) from Server B
% \begin{enumerate}
%     \item Verify that $M_A = M_B$
%     \item If $M_A = M_B$, forward to $\mathcal{Z}$
% \end{enumerate}

\extitem Upon receiving (\textbf{sid}, \textbf{SEND}, $r\oplus e_l$) from $P$, if $P$ has not executed a write request in this phase, then
    Server B executes the following:
    
    
    Upon receiving (\textbf{proof}, l(y)) from $\mathcal{F}_{ZK}^{R,R'}$, if received (\textbf{sid}, WRITE, M) from $P$:
    \begin{enumerate}
        \item add $r\oplus e_{\ell,M}$ into its database  
        \item if $count=K$, then
        \item 
        \begin{enumerate}
            \item add $r\oplus e_l$ into its database  
            \item if $count=K$, then
            \begin{enumerate}
                \item combine database with Server A's database
                \item resolve collisions
                \item order messages lexicographically as $M_B=<M_1,...,M_K>$
                \item broadcast messages to all parties
            \end{enumerate}
        \end{enumerate}
    \end{enumerate}
    Upon receiving (\textbf{sid}, WRITE, M) from $P$, if received (\textbf{proof}, l(y)) from $\mathcal{F}_{ZK}^{R,R'}$:
        \begin{enumerate}
            \item add $r\oplus e_l$ into its database  
            \item if $count=K$, then
            \begin{enumerate}
                \item combine database with Server A's database
                \item check for collisions
                \item resolve collisions
                \item order messages lexicographically as $M_B=<M_1,...,M_K>$
                \item broadcast messages to all parties
            \end{enumerate}
        \end{enumerate}


\extitem Upon receiving (\textbf{sid}, \textbf{SEND}, $r$) from $P$, if $P$ has not executed a write request in this phase, then
    Server A executes the following:
    Upon receiving (\textbf{proof}, l(y)) from $\mathcal{F}_{ZK}^{R,R'}$, if received (\textbf{sid}, WRITE, M) from $P$:
        \begin{enumerate}
            \item add $r$ into its database
            \item if $count=K$, then
            \begin{enumerate}
                \item combine database with Server B's database
                \item resolve collisions
                \item order messages lexicographically as $M_A=<M_1,...,M_K>$
                \item broadcast messages to all parties
            \end{enumerate}
        \end{enumerate}
    Upon receiving (\textbf{sid}, WRITE, M) from $P$, if received (\textbf{proof}, l(y)) from $\mathcal{F}_{ZK}^{R,R'}$:
        \begin{enumerate}
            \item XOR $r$ into its database
            \item if $count=K$, then
            \begin{enumerate}
                \item combine database with Server B's database
                \item resolve collisions
                \item order messages lexicographically as $M_A=<M_1,...,M_K>$
                \item broadcast messages to all parties
            \end{enumerate}
        \end{enumerate}
\end{tcolorbox}

\captionof{figure}{Anonymous broadcast protocol.}
\label{fig:riposte_protocol}
% \begin{tcolorbox}[enhanced, colback=white, arc=5pt, breakable]
\begin{tcolorbox}[colback=white, arc=5pt]
\noindent\emph{\underline{AE channel functionality $\mathcal{F}_{\mathcal{AEC}}(\{A,B\})$}}\\[5pt]
 Initialise a list $PendingMsg\leftarrow\emptyset$.
 
\extitem Upon receiving (\textbf{sid}, \textbf{SEND}, $M$) from P, if P is honest, then:\\
\begin{enumerate}
    \item If $\{A,B\} \setminus \{P\}$ is corrupted, then send (\textbf{sid}, \textbf{SEND}, $M$, P) to $\mathcal{S}$.
    \item  If $\{A,B\} \setminus \{P\}$ is honest, then
    \begin{itemize}
        \item Choose a random tag $\overset{\$}{\leftarrow}\{0,1\}^\lambda $.
        \item Add $(\textbf{tag}, M, P)$ to $PendingMsg$
        \item Send (\textbf{sid}, \textbf{SEND}, \textbf{tag}, $|M|$, P, \{A,B\} $\setminus$ \{P\}) to $\mathcal{S}$.
    \end{itemize}
    \item Upon receiving (\textbf{sid}, \textbf{ALLOW}, \textbf{tag}) from $\mathcal{S}$, if there is a (\textbf{tag}, $M$, P) in $PendingMsg$, then remove (\textbf{tag}, $M$, P) from $PendingMsg$ and send (\textbf{sid},\textbf{SEND},$M$) to \{A,B\}$\setminus$\{P\}
\end{enumerate}

\end{tcolorbox}
\captionof{figure}{Anonymous broadcast ideal functionality.}
\label{fig:ae_functionality}
% \begin{tcolorbox}[enhanced, colback=white, arc=5pt, breakable]
\begin{tcolorbox}[colback=white, arc=5pt]
    \noindent\emph{\underline{Non-interactive zero knowledge functionality $\mathcal{F}_{NIZK}^{R}$}}\\[5pt]
    \begin{enumerate}
        \item \textbf{Proof:} On input (\text{prove}, \textbf{sid}, $x$, $w$) from party $P$: if $\mathcal{R}(x,w)=1$ then send (\text{prove}, $P$, \textbf{sid}, $x$) to $\mathcal{S}$. Upon receiving (\text{proof}, \textbf{sid},$\pi$) from $\mathcal{S}$, store (\textbf{sid}, $x$, $w$, $\pi$) and send (\text{proof}, \textbf{sid},$\pi$) to $P$.
        \item \textbf{Verification:} On input (\text{verify}, \textbf{sid},$x$,$\pi$) from a party $V$: If (\textbf{sid},$x$,$w$,$\pi$) is stored, then return (\text{verification}, \textbf{sid}, $x$, $\pi$, $\mathcal{R}(x,w)$) to $V$. Else, send (\text{verify}, $V$, \textbf{sid}, $x$, $\pi$) to $\mathcal{S}$. Upon receiving, (\text{witness}, $\textbf{sid}$, $w$) from $\mathcal{S}$, store (\textbf{sid}, $x$, $w$, $\pi$) and return (\text{verification}, \textbf{sid}, $x$, $\pi$, $\mathcal{R}(x,w)$) to $V$.
        \item \textbf{Corruption:} When receiving (\text{corrupt}, \textbf{sid}) from $\mathcal{S}$, mark \textbf{sid} as corrupted. If there is a stored tuple (\textbf{sid}, $x$, $w$, $\pi$), then send it to $\mathcal{S}$.
    \end{enumerate}
\end{tcolorbox}
\captionof{figure}{Non-interactive zero knowledge functionality.}
\label{fig:zk_functionality}
% \begin{tcolorbox}[enhanced, colback=white, arc=5pt, breakable]
\begin{tcolorbox}[colback=white, arc=5pt]
    \noindent\emph{\underline{Broadcast functionality $\mathcal{F}_{BC}$}}\\[5pt]
    The functionality interacts with an adversary $\mathcal{S}$ and a set $\mathcal{P}=\{P_1,...,P_n\}$ of parties.
    \begin{itemize}
        \item Upon receiving (\textbf{sid}, \textbf{BROADCAST}, $M$) from $P_i$, send (\textbf{sid}, \textbf{BROADCAST}, $M$) to all parties in $\mathcal{P}$ and $\mathcal{S}$.
    \end{itemize}
    
    \end{tcolorbox}
    \captionof{figure}{Broadcast functionality $\mathcal{F}_{BC}$}
    \label{fig:bc_functionality}
\end{abstract}

%%%%%%%%%%%% DO NOT EDIT THIS PART!!! %%%%%%%%%%%%
%%%%%%%%%%%%%%%%%%%%%%%%%%%%%%%%%%%%%%%%%%%%%%%%%%
\maketitle
\tikz [remember picture, overlay] %
\node [shift={(0.5cm,-0.5cm)}] at (current page.north west) %
[anchor=north west,scale=0.7] %
{\includegraphics{CompSci_logo.pdf}};
%%%%%%%%%%%%%%%%%%%%%%%%%%%%%%%%%%%%%%%%%%%%%%%%%%
%%%%%%%%%%%%%%%%%%%%%%%%%%%%%%%%%%%%%%%%%%%%%%%%%%

% % % % % % % % % % % % % % % % % % % % %
% 			INTRODUCTION
% % % % % % % % % % % % % % % % % % % % %
\section{Introduction}
\label{sec:intro}

\section{Proof}

Cases:
\begin{enumerate}
  \item U.r. (\textbf{sid}, WRITE, $|M|$, $P$) from functionality:
  \begin{itemize}
    \item Simulate a WRITE request on behalf of $P$ where $M_{Dummy}$ is all-zeroes
    \item Generate $e_{\ell, M}$
    \item $\mathcal{F}_{ZK}$ leaks nothing. $\mathcal{F}_{AEC}$ leaks the length of the message $|M|$, so the simulator sends $|M|$ to the adversary
  \end{itemize}
  \item U.r. $<M_1,...,M_k>$ from the functionality:
  If Server A is corrupted, then
  \begin{itemize}
    \item Simulator sends a dummy message containing all zeroes over $\mathcal{F}_{AEC}$
    \item Extract the subset of honest messages from the set of all messages.
    \item Randomly assign honest messages to honest parties
    \item Generate $e_{\ell,M}$ of the corresponding party and send $r$ to Server A, then $e_{\ell,M}\oplus r$ is the share of party $P$ for Server B
  \end{itemize}
  If Server B is corrupted, then
  \begin{itemize}
    \item Simulator equivocates by sending any $r$ to Server A
    \item Extract the subset of honest messages from the set of all messages.
    \item Randomly assign honest messages to honest parties
    \item Construct a consistent $e_{\ell,M}$
    \item Send $e_{\ell,M}\oplus r$ of the corresponding party to Server B
  \end{itemize}
  \item If an adversarial message is received, then 
  \begin{itemize}
    \item Simulator reconstructs M since it controls $\mathcal{F}_{AEC}$ and every $P$ is forced to send messages across the authenticated encrypted channel because all adversaries are passive
    \item Tell the functionality to add M to the list of pending messages
  \end{itemize}
\end{enumerate}

\textbf{Case 1:} We first deal with the case where each party $P_i$ is honest. When Z provides input to $P_i$, S begins running $\pi_R$ on behalf of the honest party $P_i$. The protocol execution is easily simulated by S. In the hybrid world, the authenticated encrypted channel leaks the length of messages sent through it. In the ideal world, S emulates this by leaking the length of messages sent through itself when it plays the role of $\mathcal{F}_{AEC}$.

\textbf{Case 2:} Next we deal with the case where Server A is passively corrupted. Here $\mathcal{A}$ requests to corrupt Server A in the hybrid world and so S corrupts Server A in the ideal world. Upon receiving (\textbf{sid}, WRITE, $M_i$) from honest party $P_i$, S generates $e_{l, M_i}$ and random vector $r$. S then sends $r\oplus e_{l, M_i}$ to Server B and $r$ to Server A on behalf of $P_i$. Upon receiving (\textbf{sid}, BROADCAST, $<M_1,...,M_k$) from $\mathcal{F}_{BC}$, S extracts the subset of honest messages from $<M_1,...,M_k>$ and randomly assigns the honest messages to honest parties. S orders these messages lexicographically and forwards them to Z.

\textbf{Case 3:} The next case to consider is the case where Server B is passively corrupted. Here $\mathcal{A}$ requests to corrupt Server B in the hybrid world and so S corrupts Server A in the ideal world. Upon receiving (\textbf{sid}, WRITE, M) from honest party $P_i$, S equivocates by sending any random vector $r$ to Server A. S then generates $e_{l,M_i}$ and sends $r\oplus e_{l,M_i}$ to Server B. When the two servers combine their shares, A can send any share $g$ instead of $e_{l,M_i}$ on behalf of Server B. However, when shares are combined, $r\oplus e_{l,M_i} \oplus r$ and $g \oplus r$ share the same distribution and are thus indistinguishable. Upon receiving (\textbf{sid}, BROADCAST, $<M_1,...,M_k$) from $\mathcal{F}_{BC}$, S extracts the subset of honest messages from $<M_1,...,M_k>$ and randomly assigns the honest messages to honest parties. S orders these messages lexicographically and forwards them to Z.

\textbf{Case 4:} Lastly we consider the case where a party $P^*$ is corrupted. A requests to corrupt $P^*$ in the hybrid world so S corrupts $P^*$ in the ideal world. A sends (\textbf{sid}, WRITE, $M^*$) on behalf of $P^*$. A then continues to execute the protocol, creating shares and sending them to Server A and Server B. Since these shares are sent over $\mathcal{F}_{AEC}$, S can view these as it emulates the functionality. This allows S to reconstruct $M^*$. S can then signal its own functionality $\mathcal{F}_R^K$ to add $M^*$ to the list of pending messages.

% % % % % % % % % % % % % % % % % % % % %
% 	         BACKGROUND
% % % % % % % % % % % % % % % % % % % % %
\section{Background}
\label{sec:background}


Perhaps you want to cite the seminal paper of \citet{Turing1937}, or
prior~\cite{Goedel1931} and concurrent~\cite{Church1936} work.
% % % % % % % % % % % % % % % % % % % % %
% 			YOUR SYSTEM
% % % % % % % % % % % % % % % % % % % % %
\section{My Amazing System}
\label{sec:system}


% % % % % % % % % % % % % % % % % % % % %
% 			EVALUATION
% % % % % % % % % % % % % % % % % % % % %
\section{Evaluation}
\label{sec:evaluation}

\subsection{Experimental Setup}


\subsection{Experimental Analysis}


% Our results are summarized in~\cref{tab:table1}, and a visual representation of
% our analysis can be seen in~\cref{fig:alice}.

% %% Example of table
% \begin{table}
% \footnotesize
% \begin{tabular}{lll}
% \toprule
%          & machine A                   & machine B                           \\
% \midrule
% CPU      & Intel Core i7-9700 CPU      & 2x Intel Xeon E5-2630 v3            \\
% CPU Frequency& 3.00GHz                     & 2.40GHz                             \\
% RAM      & 16GB DDR4                   & 128GB                               \\
% OS       & Ubuntu 20.04 LTS            & Ubuntu 16.04 LTS                    \\
% Compiler & GCC 9.3                     & GCC 7.3                             \\
% libm     & v2.31                       & v2.23                               \\
% libomp   & v4.5                        & v4.5                                \\
% \bottomrule
% \end{tabular}
% \caption{This is the table caption.}
% \label{tab:table1}
% \end{table}

%% Example of figure
% \begin{figure}
%     \centering
%     \includegraphics[width=0.8\columnwidth]{alice.pdf}
%     \caption{This is the figure caption.}
%     \label{fig:alice}
% \end{figure}




% % % % % % % % % % % % % % % % % % % % %
% 			CONCLUSIONS
% % % % % % % % % % % % % % % % % % % % %
\section{Conclusions}
\label{sec:conclusions}
\lipsum[1]




% % % % % % % % % % % % % % % % % % % % %
% 			ACKNOWLEDGMENTS
% % % % % % % % % % % % % % % % % % % % %
\section*{Acknowledgments}
I would like to thank \ldots

% % % % % % % % % % % % % % % % % % % % %
% 			BIBLIOGRAPHY
% % % % % % % % % % % % % % % % % % % % %
\bibliographystyle{ACM-Reference-Format}
\bibliography{refs}
\end{document}
